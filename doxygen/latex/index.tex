Sparse L\+Inear Method (S\+L\+IM) \cite{ning2011slim} is an item-\/based top-\/N recommendation approach that combines the advantages of neighborhood-\/ and model-\/based collaborative filtering methods. It achieves state-\/of-\/the-\/art recommendation performance and has low computational requirements.

This package provides a C-\/based optimized multi-\/threaded implementation of S\+L\+IM that consists of a set of command-\/line programs and a user-\/callable library for estimating and applying S\+L\+IM models as well as an easy to use Python interface.



 \hypertarget{index_download}{}\section{Downloading S\+L\+IM}\label{index_download}
S\+L\+IM uses Git submodules to manage external dependencies. Hence, please specify the {\ttfamily -\/-\/recursive} option while cloning the repo as follow\+: 
\begin{DoxyCode}
git clone --recursive https://github.umn.edu/dminers/slim.git
\end{DoxyCode}
\hypertarget{index_installation}{}\section{Building standalone S\+L\+I\+M binary and shared library}\label{index_installation}
To build S\+L\+IM you can follow the instructions below\+:\hypertarget{index_deps}{}\subsection{Dependencies}\label{index_deps}
General dependencies for building slim are\+: gcc, cmake, build-\/essential. In Ubuntu systems these can be obtained from the apt package manager (e.\+g., apt-\/get install cmake, etc)

\begin{DoxyVerb}sudo apt-get install build-essential
sudo apt-get install cmake
\end{DoxyVerb}
\hypertarget{index_without_mkl}{}\subsection{Building and installing S\+L\+IM}\label{index_without_mkl}
In order to build S\+L\+IM, first build G\+Klib by running\+:

\begin{DoxyVerb}cd lib/GKlib
make config openmp=set
make
cd ../../
\end{DoxyVerb}


After building G\+Klib, you can build and install S\+L\+IM by running\+:

\begin{DoxyVerb}make config shared=1 cc=gcc cxx=gcc prefix=~/.local
make install
\end{DoxyVerb}
\hypertarget{index_mkl_build}{}\subsubsection{Building S\+L\+I\+M with Intel\textquotesingle{}s M\+K\+L support (optional)}\label{index_mkl_build}
In order to use S\+L\+IM\textquotesingle{}s A\+D\+MM solver, you will need to install \href{https://software.intel.com/en-us/mkl}{\tt Intel\textquotesingle{}s M\+KL library}.

For Ubuntu machines on which you have {\ttfamily sudo} privilages, we provided the {\ttfamily depmkl.\+sh} script that automates the process of obtaining and installing M\+KL, which can be used as follows\+:

\begin{DoxyVerb}bash depmkl.sh
source ~/.bashrc 
\end{DoxyVerb}


For machines on which you do not have {\ttfamily sudo} privilages, you should download the M\+KL tarball from \href{https://software.intel.com/en-us/mkl}{\tt Intel\textquotesingle{}s website} and then install it locally using the {\ttfamily install.\+sh} script they provide. After installing it you should add {\ttfamily your-\/path-\/to-\/intel/intel/mkl/bin/mklvars.\+sh intel64}in your bashrc and run {\ttfamily source $\sim$/.bashrc}.

You can build and install S\+L\+IM with M\+KL support by running\+:

\begin{DoxyVerb}make config shared=1 cc=gcc cxx=gcc with_mkl=1 prefix=~/.local
make install
\end{DoxyVerb}


Note that S\+L\+IM\textquotesingle{}s A\+D\+MM solver usually outperforms the default optimizer included in S\+L\+IM when the number of items in the dataset is relatively small compared to the number of users and the number of non-\/zeros in the dataset is large.\hypertarget{index_pythonpackage}{}\section{Python Package installation}\label{index_pythonpackage}
The Python package is located at {\ttfamily python-\/package/}. The installation of python-\/package requires Python {\ttfamily distutils} module and is often part of the core Python package or can be installed using a package manager, e.\+g., in Debian use

\begin{DoxyVerb}sudo apt-get install python-setuptools
\end{DoxyVerb}


After building the S\+L\+IM library, follow one of the following steps to install the python-\/package\+:


\begin{DoxyEnumerate}
\item Install python-\/package system-\/wide (this requires sudo priveleges)\+: \begin{DoxyVerb}cd python-package
sudo python setup.py install
\end{DoxyVerb}

\item Install python-\/package only for the current users (without sudo priveleges)\+: \begin{DoxyVerb}cd python-package
python setup.py install --user
\end{DoxyVerb}

\end{DoxyEnumerate}\hypertarget{index_getting_started}{}\section{Getting Started}\label{index_getting_started}
Here are some examples to quickly try out S\+L\+IM on the sample datasets that are provided with S\+L\+IM.\hypertarget{index_python_getting_started}{}\subsection{Python}\label{index_python_getting_started}

\begin{DoxyCode}
\textcolor{keyword}{import} pandas \textcolor{keyword}{as} pd
\textcolor{keyword}{from} SLIM \textcolor{keyword}{import} SLIM, SLIMatrix

\textcolor{comment}{#read training data stored as triplets <user> <item> <rating>}
traindata = pd.read\_csv(\textcolor{stringliteral}{'../test/AutomotiveTrain.ijv'}, delimiter = \textcolor{stringliteral}{' '}, header=\textcolor{keywordtype}{None})
trainmat = SLIMatrix(traindata)

\textcolor{comment}{#set up parameters to learn model, e.g., use Coordinate Descent with L1 and L2}
\textcolor{comment}{#regularization}
params = \{\textcolor{stringliteral}{'algo'}:\textcolor{stringliteral}{'cd'}, \textcolor{stringliteral}{'nthreads'}:2, \textcolor{stringliteral}{'l1r'}:1., \textcolor{stringliteral}{'l2r'}:1.\}

\textcolor{comment}{#learn the model using training data and desired parameters}
model = SLIM()
model.train(params, trnmat)

\textcolor{comment}{#read test data having candidate items for users}
testdata = pd.read\_csv(\textcolor{stringliteral}{'../test/AutomotiveTest.ijv'}, delimiter = \textcolor{stringliteral}{' '}, header=\textcolor{keywordtype}{None})
\textcolor{comment}{#NOTE: model object is passed as an argument while generating test matrix}
testmat = SLIMatrix(testdata, model)

\textcolor{comment}{#generate top-10 recommendations}
prediction\_res = model.predict(testmat, nrcmds=10, outfile = \textcolor{stringliteral}{'output.txt'})

\textcolor{comment}{#dump the model to files on disk}
model.save\_model(modelfname=\textcolor{stringliteral}{'model.csr'}, \textcolor{comment}{# filename to save the model as a csr matrix}
                 mapfname=\textcolor{stringliteral}{'map.csr'} \textcolor{comment}{# filename to save the item map}
                )

\textcolor{comment}{#load the model from from disk}
model\_new = SLIM()
model\_new.load\_model(modelfname=\textcolor{stringliteral}{'model.csr'}, \textcolor{comment}{# filename of the model}
                 mapfname=\textcolor{stringliteral}{'map.csr'} \textcolor{comment}{# filename of the item map}
                )
\end{DoxyCode}


The users can also refer to the python notebook {\ttfamily User\+Guide.\+ipynb} located at {\ttfamily ./python-\/package/\+User\+Guide.ipynb} for more examples on using the python api.\hypertarget{index_standalone_getting_Started}{}\subsection{Standalone}\label{index_standalone_getting_Started}
S\+L\+IM can also be used standalone by running the built binaries located under {\ttfamily ./build} directory. The {\ttfamily slim\+\_\+learn} and {\ttfamily slim\+\_\+predict} program can be used to learn the model and predict using an existing model, respectively. The usage of {\ttfamily slim\+\_\+learn} is generally as follow\+:

\begin{DoxyVerb}slim_learn [options] train-file [output-model-file] 
\end{DoxyVerb}


In above command, {\ttfamily train-\/file} refers to the rating matrix stored in a file on disk in sparse format (by default it expects C\+SR format) and {\ttfamily output-\/model-\/file} is the name of the file that will be used to save the learned model. The available options will be described later in detail. Following shows an example of using the {\ttfamily slim\+\_\+learn} program\+:

\begin{DoxyVerb}./build/Linux-x86_64/src/programs/slim_learn test/ml100k-train.csr model_output.slim
\end{DoxyVerb}


Similarly, the recommendations can be generated by using the program {\ttfamily slim\+\_\+predict} and the usage of {\ttfamily slim\+\_\+predict} is as follow\+:

\begin{DoxyVerb}slim_predict [options] model-file train-file [test-file]
\end{DoxyVerb}


In above command, {\ttfamily model-\/file} refers to the S\+L\+IM model saved in a file, {\ttfamily train-\/file} refers to the training data or past historical data used to generate the model and {\ttfamily test-\/file} is the test data containing hidden ratings for the users.\hypertarget{index_parameters}{}\section{S\+L\+I\+M parameters}\label{index_parameters}
The following optional parameters can be provided to the {\ttfamily slim\+\_\+learn} program.
\begin{DoxyItemize}
\item ifmt \mbox{[}default=csr\mbox{]}
\begin{DoxyItemize}
\item Specifies the format of the input file. Available options are\+: csr -\/ C\+SR format \mbox{[}default\mbox{]}. csrnv -\/ C\+SR format without ratings. cluto -\/ Format used by C\+L\+U\+TO. ijv -\/ One (row\#, col\#, val) per line.
\end{DoxyItemize}
\item binarize \mbox{[}default=0\mbox{]}
\begin{DoxyItemize}
\item Specifies that the ratings should be binarized.
\end{DoxyItemize}
\item l1r \mbox{[}default=1.\+0\mbox{]}
\begin{DoxyItemize}
\item Specifies the L1 regularization parameter.
\end{DoxyItemize}
\item ipmdlfile \mbox{[}default=\char`\"{}\char`\"{}\mbox{]}
\begin{DoxyItemize}
\item Specifies the file used to initilize the model.
\end{DoxyItemize}
\item l2r \mbox{[}default=1.\+0\mbox{]}
\begin{DoxyItemize}
\item Specifies the L2 regularization parameter.
\end{DoxyItemize}
\item nnbrs \mbox{[}default=0\mbox{]}
\begin{DoxyItemize}
\item Selects f\+S\+L\+IM model and specifies the number of item nearest neighbors to be used. Specifying few neighbors will speed-\/up the model learning but it may lower the accuracy. The parameter {\itshape simtype} sets the measurement of similarity. This package supports three similarity measurements, Jaccard similarity ("jac"), Cosine similarity ("cos"), and inner product ("dotp"). The default value for {\itshape simtype} is "cos". A f\+S\+L\+IM model can be used in the same way with a S\+L\+IM model. Note that, a f\+S\+L\+IM model can only be trained using coordinate descent.
\end{DoxyItemize}
\item simtype \mbox{[}default=cos\mbox{]}
\begin{DoxyItemize}
\item Specifies the similarity function for determining the neighbors. Available options are\+:
\begin{DoxyItemize}
\item cos -\/ cosine similarity \mbox{[}default\mbox{]}.
\item jac -\/ extended Jacquard similarit.
\item dotp -\/ dot-\/product similarity.
\end{DoxyItemize}
\end{DoxyItemize}
\item algo \mbox{[}default=cd\mbox{]}
\begin{DoxyItemize}
\item Specifies the optimization algorithms for learning the model. Available options are\+:
\begin{DoxyItemize}
\item admm -\/ A\+D\+MM.
\item cd -\/ Coordinate Descent.
\end{DoxyItemize}
\end{DoxyItemize}
\item opt\+Tol \mbox{[}default=1e-\/7\mbox{]}
\begin{DoxyItemize}
\item Specifies the threshold used during optimization for termination.
\end{DoxyItemize}
\item niters \mbox{[}default=10000\mbox{]}
\begin{DoxyItemize}
\item Specifies the maximum number of allowed optimization iterations.
\end{DoxyItemize}
\item nthreads
\begin{DoxyItemize}
\item Specifies the number of threads to be used for estimation. Default value is maximum number of threads available on the machine.
\end{DoxyItemize}
\item dbglvl
\begin{DoxyItemize}
\item Specifies the debug level. The default value turns on info and timing.
\end{DoxyItemize}
\item help
\begin{DoxyItemize}
\item Prints the parameter details.
\end{DoxyItemize}
\end{DoxyItemize}\hypertarget{index_Citing}{}\section{Citing}\label{index_Citing}
If you use any part of this library in your research, please cite it using the following Bib\+Tex entry\+:

\begin{DoxyVerb}@online{slim,
  title = {{SLIM}: Sparse LInear Model library},
  author = {Ning, Xia and Nikolakopoulos, Athanasios N. and Shui, Zeren and Sharma, Mohit and Karypis, George},
  url = {https://github.com/KarypisLab/SLIM},
  publisher = {GitHub},
  journal = {GitHub Repository},
  year = {2019},
}
\end{DoxyVerb}
\hypertarget{index_contact}{}\section{Credits \& Contact Information}\label{index_contact}
S\+L\+IM was written by George Karypis with contributions by Xia Ning, Athanasios N. Nikolakopoulos, Zeren Shui and Mohit Sharma.

If you encounter any problems or have any suggestions, please contact George Karypis at \href{mailto:karypis@cs.umn.edu}{\tt karypis@cs.\+umn.\+edu}.\hypertarget{index_license}{}\section{Copyright \& License Notice}\label{index_license}
Copyright 2019, Regents of the University of Minnesota

Licensed under the Apache License, Version 2.\+0 (the \char`\"{}\+License\char`\"{}); you may not use this file except in compliance with the License. You may obtain a copy of the License at

\href{http://www.apache.org/licenses/LICENSE-2.0}{\tt http\+://www.\+apache.\+org/licenses/\+L\+I\+C\+E\+N\+S\+E-\/2.\+0}

Unless required by applicable law or agreed to in writing, software distributed under the License is distributed on an \char`\"{}\+A\+S I\+S\char`\"{} B\+A\+S\+IS, W\+I\+T\+H\+O\+UT W\+A\+R\+R\+A\+N\+T\+I\+ES OR C\+O\+N\+D\+I\+T\+I\+O\+NS OF A\+NY K\+I\+ND, either express or implied. See the License for the specific language governing permissions and limitations under the License. 